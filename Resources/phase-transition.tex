\documentclass[a4paper]{article} 
\usepackage{tcolorbox}
\usepackage{amsmath,amsfonts,amsthm}
\tcbuselibrary{skins}
\usepackage{graphicx}
\usepackage{hyperref}

\makeatletter
\def\cornell{\@ifnextchar[{\@with}{\@without}}
\def\@with[#1]#2#3{
\begin{tcolorbox}[enhanced,colback=white!15,colframe=white,colupper=gray]
\begin{tcolorbox}[enhanced,colback=gray,colframe=black,fonttitle=\large\bfseries\sffamily,sidebyside, nobeforeafter,colupper=black,
righthand width=14cm,
opacityframe=0,opacityback=0.3,opacitybacktitle=1, opacitytext=1,
segmentation style={black!55,solid,opacity=0,line width=1pt},
title=#1
]
%\begin{tcolorbox}[colback=red!05,colframe=red!25,sidebyside align=top,
%width=\textwidth,nobeforeafter]#2\end{tcolorbox}%
%\tcblower
%\sffamily
%\begin{tcolorbox}[colback=blue!05,width=\textwidth]
% #3
%\end{tcolorbox}
\Huge {\bf  #2}
\tcblower
#3
\end{tcolorbox}
\end{tcolorbox}
}
\makeatother

\title{
\vspace{-3em}
\begin{tcolorbox}
\Huge\sffamily AMA4004 Statistical mechanics: Understanding phase transitions
\end{tcolorbox}
\vspace{-3em}
}

\date{}

%\usepackage{background}
%\SetBgScale{1}
%\SetBgAngle{0}
%\SetBgColor{red}
%\SetBgContents{\rule[0em]{4pt}{\textheight}}
%\SetBgHshift{-2.3cm}
%\SetBgVshift{0cm}
\usepackage{lipsum}% just to generate filler text for the example
\usepackage[margin=1.5cm]{geometry}
\usepackage{lipsum}% just to generate dummy text for the example


\begin{document}
\maketitle

{\bf The following problem should be solved by writing a computer program.  This is a hard project and thus, if it is submitted for the portfolio, there is the pontential to get marks of 9/10 out of 12 for it.}

\vpsace{1cm}

We can introduce a very simple model for the transition between a fluid and a gas phase of some substance as follows.  $N$ indistinghisable particles are
allowed to move freedly on a two dimensional squarre lattice with $V$ sites (where $1 \ll N \ll V$).  The particles are subject to a short ranged attractive
interaction, which makes the energy of the system $E$ equal to $-\epsilon$ times the number of particle pairs of neighbouring lattice sites.  The kinetic energy
of the particles is completely neglected in this model. In the fluid phase the particles condense into a single connected block, while in the gas phase the particles
are distributed randomly on the lattice.

\begin{enumerate}
\item Write a program to generate all the possible microstates for a version of this model with 16 lattice sites and 4 particles. Calculate the energies of all these microstates and hence draw a graph showing how the ensemble average of the energy changes with temperature. Discuss the behavior you observe. 
\item Calculate the histogram showing the most likely configurations that this system will adopt at two distinct temperatures 
\item Give a justification for why the free energy of the fluid and gas phases are -$2N\epsilon$ and $-k_B TN \ln \frac{V}{N}$ respectively.  Try to determine the temperature at which the transition between the fluid and the gas phase takes place. 
\item Discuss how this system would behave if the interactions between particles were turned off.  
\end{enumerate}

\end{document}
