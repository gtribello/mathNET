\documentclass[paper=a4, fontsize=11pt]{scrartcl}
\usepackage[T1]{fontenc}
%\usepackage{fourier}

\usepackage[english]{babel}
\usepackage[protrusion=true,expansion=true]{microtype}	
\usepackage{amsmath,amsfonts,amsthm} % Math packages
\usepackage[pdftex]{graphicx}	
\usepackage{url}
\usepackage{makecell,pict2e}
\usepackage{comment}
\usepackage{color}

\renewcommand\theadfont{\large}
\newcommand{\gt}[1]{{\color{blue}#1}}
\newcommand{\red}[1]{{\color{red}#1}}

\renewcommand\theadfont{\large}

%%% Custom sectioning
\usepackage{sectsty}
\allsectionsfont{\normalfont\scshape}

\includecomment{answers}
% \excludecomment{answers}

\usepackage[margin=2cm]{geometry}
\setlength{\topmargin}{-2.cm}
\setlength{\headheight}{1cm}

%%% Custom headers/footers (fancyhdr package)
\usepackage{fancyhdr}
\pagestyle{fancyplain}
\fancyhead{}
\fancyfoot[L]{}
\fancyfoot[C]{}
\fancyfoot[R]{\thepage}
\renewcommand{\headrulewidth}{0pt}
\renewcommand{\footrulewidth}{0pt}
\setlength{\headheight}{13.6pt}

%%% Equation and float numbering
\numberwithin{equation}{section}
\numberwithin{figure}{section}
\numberwithin{table}{section}


%%% Maketitle metadata
\newcommand{\horrule}[1]{\rule{\linewidth}{#1}}
\newcommand{\vek}[1]{\mbox{\boldmath $  #1$}}
\newcommand{\ex}[1]{\ensuremath {\mathbb{E}} \left[ #1 \right]}
\newcommand{\var}[1]{\ensuremath{{\mathrm var}\left[ #1 \right]}}

\title{\usefont{OT1}{bch}{b}{n} \normalfont \normalsize \textsc{SOR3012:
Stochastic Processes} \\ [25pt] \horrule{0.5pt} \\[0.4cm] 
\huge Poisson Process Project \\
\horrule{2pt} \\[0.25cm]
}
\author{ \normalfont
\normalsize
        Gareth Tribello \\[-3pt] \normalsize
        \today
}
\date{}

\begin{document}
\maketitle

For this project you must produce a {\bfseries three page} set of notes on the Poisson Process.  You should prepare your report as an ipython notebook and within it you should present:

\begin{itemize}
 \item You should discuss phenomena that can be modelled using a Poisson process and you should discuss the assumptions that are made when this model is used to model these phenomena

 \item You should provide information on the transition graph, the jump rate matrix and the classification of the states in the Poisson process.  You should then discuss how equations for the probabilities for being in each of the states of the chain are derived by solving the Kolmogorov equations. 

 \item You should discuss how one can perform numerical simulations of a Poisson process and include suitable python scripts for performing such simulations.  You should ensure that you carefully test the results from your numerical simulations against the results that are to be expected given the predictions of the analytical model.

 \item You should discuss at least two ways that the assumptions of Markovianity can be relaxed in the Poisson process and what new phenomena can be modelled given these results.

 \item You should discuss how to fit the parameters of a Poisson and a Compound poisson process from some data set that describes the times at which a sequence of events occurs. 
\end{itemize}

You should also explain any results you have used on series expansions or special integrals within your notes and provide the reader with links explaining where they can find material that explains 
these ideas.  These links might be pointers to particular websites, pointers to notes from modules that you have studied at Queen's or pointers to websites.  You {\bfseries should not} get bogged down in 
explaining what a Markov chain is, how the Kolmogorov equations are derived and so on. You can assume that the readers of your report will have studied what you have in SOR3012 and that they will thus
be familiar with these details.

Your report should be output as a pdf from the python notebook software and handed in through the online submission system.

You are strongly encouraged to discuss your report with your colleages.  One thing you might consider doing is using the markscheme below to mark each others report.  You can then give each other 
constructive feedback on what has been done well and what has been done poorly.

On my website you will find an example report on the exponential random variable so that you have some guidance as to what I expect.

\section{Markscheme}

\begin{center}
\begin{tabular}{ l | c | c }
Description & Marks available & Final mark \\ \hline
Discussion of what Poisson process can be used to model & 2 & \\
Transition graphs & 1 & \\
Jump rate matrices & 1 & \\
Derivations using Kolmogorov forward equations & 2 & \\
Numerical simulations of poisson process & 1 & \\
Test of numerical simulations against analytical results & 2 & \\
Discussion of how assumption of Markovianity can be relaxed and advantages this confers & 2 & \\
Discussion of how to fit data to poisson process and numerical simulations of fitting & 3 & \\
References were stated & 1 
\end{tabular}
\end{center}


\end{document}
