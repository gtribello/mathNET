\documentclass[a4paper]{article} 
\usepackage{tcolorbox}
\usepackage{amsmath,amsfonts,amsthm}
\tcbuselibrary{skins}
\usepackage{graphicx}
\usepackage{hyperref}

\makeatletter
\def\cornell{\@ifnextchar[{\@with}{\@without}}
\def\@with[#1]#2#3{
\begin{tcolorbox}[enhanced,colback=white!15,colframe=white,colupper=gray]
\begin{tcolorbox}[enhanced,colback=gray,colframe=black,fonttitle=\large\bfseries\sffamily,sidebyside, nobeforeafter,colupper=black,
righthand width=14cm,
opacityframe=0,opacityback=0.3,opacitybacktitle=1, opacitytext=1,
segmentation style={black!55,solid,opacity=0,line width=1pt},
title=#1
]
%\begin{tcolorbox}[colback=red!05,colframe=red!25,sidebyside align=top,
%width=\textwidth,nobeforeafter]#2\end{tcolorbox}%
%\tcblower
%\sffamily
%\begin{tcolorbox}[colback=blue!05,width=\textwidth]
% #3
%\end{tcolorbox}
\Huge {\bf  #2}
\tcblower
#3
\end{tcolorbox}
\end{tcolorbox}
}
\makeatother

\title{
\vspace{-3em}
\begin{tcolorbox}
\Huge\sffamily AMA4004 Statistical mechanics: Adsorption
\end{tcolorbox}
\vspace{-3em}
}

\date{}

%\usepackage{background}
%\SetBgScale{1}
%\SetBgAngle{0}
%\SetBgColor{red}
%\SetBgContents{\rule[0em]{4pt}{\textheight}}
%\SetBgHshift{-2.3cm}
%\SetBgVshift{0cm}
\usepackage{lipsum}% just to generate filler text for the example
\usepackage[margin=1.5cm]{geometry}
\usepackage{lipsum}% just to generate dummy text for the example


\begin{document}
\maketitle

In order to do this exercise you will need to work through the following topics:

\begin{itemize}
\item \href{http://gtribello.github.io/mathNET/CANONICAL\_ENSEMBLE.html}{http://gtribello.github.io/mathNET/CANONICAL\_ENSEMBLE.html}
\item \href{http://gtribello.github.io/mathNET/GRAND\_CANONICAL\_ENSEMBLE.html}{{http://gtribello.github.io/mathNET/GRAND\_CANONICAL\_ENSEMBLE.html}
\item \href{http://gtribello.github.io/mathNET/ADSORPTION.html}{http://gtribello.github.io/mathNET/ADSORPTION.html}
\end{itemize}

The Haber process is the industrial process that is used to make the majority of the ammonia that is used in fertilizers and explosives.  In one of the steps in this process iron is used as a heterogeneous catalyst and to speed up the rate of reaction 
of the nitrogen and hydrogen gases.  Write an essay that explains the derivation of a model that can be used to describe the adsorption of these gases on the iron catalyst.  Within your essay you should:   

\begin{enumerate}
\item Discuss what a heterogeneous catalyst is and how a heterogeneous catalyst differs from a homogeneous catalyst.
\item Discuss the derivatition of the Langmuir adsorption isotherm, plot the isotherm for various values of the temperature and discuss the approximations and limitations of this model 
\item {\bf (Hard)} Generalize the Langmuir adsorption isotherm and hence obtain a model that can be used to describe the adsorption of two different gasses on a solid surface.
\end{enumerate}

\end{document}
