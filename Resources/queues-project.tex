\documentclass[paper=a4, fontsize=11pt]{scrartcl}
\usepackage[T1]{fontenc}
%\usepackage{fourier}

\usepackage[english]{babel}
\usepackage[protrusion=true,expansion=true]{microtype}	
\usepackage{amsmath,amsfonts,amsthm} % Math packages
\usepackage[pdftex]{graphicx}	
\usepackage{url}
\usepackage{makecell,pict2e}
\usepackage{comment}
\usepackage{color}

\renewcommand\theadfont{\large}
\newcommand{\gt}[1]{{\color{blue}#1}}
\newcommand{\red}[1]{{\color{red}#1}}

\renewcommand\theadfont{\large}

%%% Custom sectioning
\usepackage{sectsty}
\allsectionsfont{\normalfont\scshape}

\includecomment{answers}
% \excludecomment{answers}

\usepackage[margin=2cm]{geometry}
\setlength{\topmargin}{-2.cm}
\setlength{\headheight}{1cm}

%%% Custom headers/footers (fancyhdr package)
\usepackage{fancyhdr}
\pagestyle{fancyplain}
\fancyhead{}
\fancyfoot[L]{}
\fancyfoot[C]{}
\fancyfoot[R]{\thepage}
\renewcommand{\headrulewidth}{0pt}
\renewcommand{\footrulewidth}{0pt}
\setlength{\headheight}{13.6pt}

%%% Equation and float numbering
\numberwithin{equation}{section}
\numberwithin{figure}{section}
\numberwithin{table}{section}


%%% Maketitle metadata
\newcommand{\horrule}[1]{\rule{\linewidth}{#1}}
\newcommand{\vek}[1]{\mbox{\boldmath $  #1$}}
\newcommand{\ex}[1]{\ensuremath {\mathbb{E}} \left[ #1 \right]}
\newcommand{\var}[1]{\ensuremath{{\rm var}\left[ #1 \right]}}

\title{\usefont{OT1}{bch}{b}{n} \normalfont \normalsize \textsc{SOR3012:
Stochastic Processes} \\ [25pt] \horrule{0.5pt} \\[0.4cm] 
\huge M/M/1 queue \\
\horrule{2pt} \\[0.25cm]
}
\author{ \normalfont
\normalsize
        Gareth Tribello \\[-3pt] \normalsize
        \today
}
\date{}

\begin{document}
\maketitle

For this project you must produce a {\bf three page} set of notes on the M/M/1 queue.  You should prepare your report as an ipython notebook and within it you should present:

\begin{itemize}
\item You should discuss how a model can be constructed by using the theory of continuous time Markov chains.  In this discussion you should talk about the distinction between infinite 
capacity and finite capacity queues.  You should also include suitable transition graphs and jump rate matrices.

\item You should derive an expression for the stationary distribution of both the finite capacity and infinite capaciy queues.

\item You should write a piece of code that can be used to simulate the M/M/1 queue (see instructions below).  You should ensure that you carefully test the predictions of your numerical
model against the predictions of the true analytical model.  

\item You should discuss how the M/M/1 queue can be extended and how one can construct queues that have multiple servers.  You may want to include a numerical model of a queue with multiple servers
and you may also want to include an analytical solution to this model.

\item You should discuss models of queues in which the assumptions of Markovianity are relaxed.  What are the benefits of using such models and what additional difficulties does using such models entail.

\end{itemize}

You should also explain any results you have used on series expansions or special integrals within your notes and provide the reader with links explaining where they can find material that explains 
these ideas.  These links might be pointers to particular websites, pointers to notes from modules that you have studied at Queen's or pointers to websites.  You {\bf should not} get bogged down in 
explaining what a Markov chain is, how the Kolmogorov equations are derived and so on. You can assume that the readers of your report will have studied what you have in SOR3012 and that they will thus
be familiar with these details.

Your report should be output as a pdf from the python notebook software and handed in through the online submission system.

You are strongly encouraged to discuss your report with your colleages.  One thing you might consider doing is using the markscheme below to mark each others report.  You can then give each other 
constructive feedback on what has been done well and what has been done poorly.

On my website you will find an example report on the exponential random variable so that you have some guidance as to what I expect.

\section{Markscheme}

\begin{center}
\begin{tabular}{ l | c | c }
Description & Marks available & Final mark \\ \hline
Discussion of the components of the queue & 1 & \\
Discussion of the difference between finite and infinite capacity queues & 1 & \\
Transition graphs & 1 & \\
Jump rate matrices & 1 & \\
Derivations of the stationary distribution & 2 & \\
Numerical simulations of a queue & 2 & \\
Test of numerical simulations against analytical results & 2 & \\
Discussion of queues with more than one server & 2 & \\
Discussion of how assumption of Markovianity can be relaxed and new models can be generated & 2 & \\
References were stated & 1 
\end{tabular}
\end{center}

\section{Numerical algorithms for queues}

In all the exercises we have performed prior to this one we have simulated the random events in "real time."  I mean by this 
that the order in which the random numbers that underpin our phenomenon were generated was the same as the order in which 
those events would happen in actuality.  We cannot perform simulations of a queue in the same way.  Instead we have to generate
two lists of random numbers at the outset.  These lists contain:

\begin{enumerate}
\item The time at which each person who enters our queue arrives at the queue.  You should be able to generate this list easily using what you have learnt elsewhere.

\item The length of time each person who enters the queue spends being served.  Again this list should be easy to generate given what you know about generating particular kinds of random variables.
\end{enumerate}

From these two lists you should then be able to determine the time at which each person enters service and the time at which they leave service by simple arithmatic.  You can thus determine the total amount of time each person spends in the queue.


\end{document}
