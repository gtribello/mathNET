\documentclass[paper=a4, fontsize=11pt]{scrartcl}
\usepackage[T1]{fontenc}
%\usepackage{fourier}

\usepackage[english]{babel}
\usepackage[protrusion=true,expansion=true]{microtype}	
\usepackage{amsmath,amsfonts,amsthm} % Math packages
\usepackage[pdftex]{graphicx}	
\usepackage{url}
\usepackage{makecell,pict2e}
\usepackage{comment}
\usepackage{color}

\renewcommand\theadfont{\large}
\newcommand{\gt}[1]{{\color{blue}#1}}
\newcommand{\red}[1]{{\color{red}#1}}

\renewcommand\theadfont{\large}

%%% Custom sectioning
\usepackage{sectsty}
\allsectionsfont{\normalfont\scshape}

\includecomment{answers}
% \excludecomment{answers}

\usepackage[margin=2cm]{geometry}
\setlength{\topmargin}{-2.cm}
\setlength{\headheight}{1cm}

%%% Custom headers/footers (fancyhdr package)
\usepackage{fancyhdr}
\pagestyle{fancyplain}
\fancyhead{}
\fancyfoot[L]{}
\fancyfoot[C]{}
\fancyfoot[R]{\thepage}
\renewcommand{\headrulewidth}{0pt}
\renewcommand{\footrulewidth}{0pt}
\setlength{\headheight}{13.6pt}

%%% Equation and float numbering
\numberwithin{equation}{section}
\numberwithin{figure}{section}
\numberwithin{table}{section}


%%% Maketitle metadata
\newcommand{\horrule}[1]{\rule{\linewidth}{#1}}
\newcommand{\vek}[1]{\mbox{\boldmath $  #1$}}
\newcommand{\ex}[1]{\ensuremath {\mathbb{E}} \left[ #1 \right]}
\newcommand{\var}[1]{\ensuremath{{\mathrm var}\left[ #1 \right]}}

\title{\usefont{OT1}{bch}{b}{n} \normalfont \normalsize \textsc{AMA4004:
Statistical Mechanics} \\ [25pt] \horrule{0.5pt} \\[0.4cm] 
\huge 1D-open Ising model project \\
\horrule{2pt} \\[0.25cm]
}
\author{ \normalfont
\normalsize
        Gareth Tribello \\[-3pt] \normalsize
        \today
}
\date{}

\begin{document}
\maketitle

For this project you must produce a {\bfseries three page} set of notes on the 1D-open Ising model.  You should prepare your report as an ipython notebook and within 
it you should present:

\begin{itemize}
 \item A statement of the model Hamiltonian for this particular system.
 \item A python function that evaluates this Hamiltonian from the microscopic coordinates of the system. 
 \item A derivation that arrives at an analytic expression for the canonical partition function for this particular system.
 \item A derivation that arrives at an analytic expression for the ensemble average of the energy as a function of temperature for this particular system.
 \item A python function that evalutes the ensemble average of the energy by enumerating over all the possible microstates the system can occupy.  
 \item A graph showing how the ensemble average of the energy changes with temperature.  This graph should be accompanied by some discussion on what your graph shows.
 \item A python function that evaluates the equilibrium distribution for the total magnetization of the system.
 \item Plots of the equilibrium distribution for the total magnetization evaluated at two temperatures.  There should be a suitable discussion of what these plots show to accompany these graphs.
 \item A python function that evaluates the spin-spin correlation function for this system exactly.
 \item Plots of the spin-spin correlation function evaluated at two temperatures.  Again there should be some discussion of what these plots show and how the behavior of the spin-spin correlation function for your system differs from the behavior observed for the lattice gas.
 \item A short discussion outlining why we can only use the algorithms discussed in this report to calculate partition functions and ensemble averages when the system size is small.
 \item A short list of references that you used to compile the report.
\end{itemize}

You should explain any results you have used from linear algebra within your notes and provide the reader with links explaining where they can find material that explains 
these ideas.  These links might be pointers to particular websites, or pointers to notes from modules that you have studied at Queen's or pointers to books.  You do not need 
to derive these theorems, however.

You should not get bogged down in the abstract details of statistical mechanics, the way an ensemble average is calculated and so on.  You can assume that the readers of your 
report will know these details.

You are strongly encouraged to discuss your report with your colleages.  One thing you might concisder doing is using the markscheme below to mark each others report.  
You can then give each other constructive feedback on what has been done well and what has been done poorly.

On my website you will find an example report on lattice gases so that you have some guidance as to what I expect.

\section{Markscheme}

\begin{center}
\begin{tabular}{ l | c | c }
Description & Marks available & Final mark \\ \hline
Hamiltonian was stated & 1 & \\
Function for evaluating Hamiltonian was provided & 1 & \\
Analytic expression for partition function & 1 & \\
Derivation of analytic expression for partition function & 1 & \\
Python function for evaluating ensemble average of energy & 1 & \\
Graphs of ensemble average as a function of temperature & 1 & \\
Discussion of behavior of ensemble average graphs & 1 & \\
Python function for evaluating histogram & 1 & \\
Plots of histograms & 1 & \\
Discussion of behavior of histograms & 1 & \\
Python function for evaluating spin-spin correlation & 1 & \\
Plots of spin-spin correlation function & 1 & \\
Discussion of spin-spin correlation functions & 1 & \\
Discussion of limitation of explicit enumeration of states & 1 & \\
References were stated & 1 & 
\end{tabular}
\end{center}


\end{document}
