\documentclass[a4paper]{article} 
\usepackage{tcolorbox}
\tcbuselibrary{skins}

\title{
\vspace{-3em}
\begin{tcolorbox}
\Huge\sffamily Statistical mechanics lecture 1   
\end{tcolorbox}
\vspace{-3em}
}

\date{}

\usepackage{background}
\SetBgScale{1}
\SetBgAngle{0}
\SetBgColor{red}
\SetBgContents{\rule[0em]{4pt}{\textheight}}
\SetBgHshift{-2.3cm}
\SetBgVshift{0cm}
\usepackage{lipsum}% just to generate filler text for the example
\usepackage[margin=1.5cm]{geometry}
\usepackage{lipsum}% just to generate dummy text for the example


%\url{http://tex.stackexchange.com/a/314/86}

\makeatletter
\def\cornell{\@ifnextchar[{\@with}{\@without}}
\def\@with[#1]#2#3{
\begin{tcolorbox}[enhanced,colback=white!15,colframe=white,colupper=gray]
\begin{tcolorbox}[enhanced,colback=gray,colframe=black,fonttitle=\large\bfseries\sffamily,sidebyside=false, nobeforeafter,colupper=black,sidebyside align=top,
opacityframe=0,opacityback=0.3,opacitybacktitle=1, opacitytext=1,
segmentation style={black!55,solid,opacity=0,line width=1pt},
title=#1
]
%\begin{tcolorbox}[colback=red!05,colframe=red!25,sidebyside align=top,
%width=\textwidth,nobeforeafter]#2\end{tcolorbox}%
%\tcblower
%\sffamily
%\begin{tcolorbox}[colback=blue!05,width=\textwidth]
#3
%\end{tcolorbox}
\end{tcolorbox}
\end{tcolorbox}
}
\def\@without#1#2{
\begin{tcolorbox}[enhanced,colback=white!15,colframe=white,fonttitle=\bfseries,sidebyside=true, nobeforeafter,before=\vfil,after=\vfil,colupper=blue,sidebyside align=top, lefthand width=.20\textwidth,
opacityframe=0,opacityback=0,opacitybacktitle=0, opacitytext=1,
segmentation style={black!55,solid,opacity=0,line width=3pt}
]

\begin{tcolorbox}[colback=red!05,colframe=red!25,sidebyside align=top,
width=\textwidth,nobeforeafter]#1\end{tcolorbox}%
\tcblower
\sffamily
\begin{tcolorbox}[colback=blue!05,colframe=blue!10,width=1\textwidth,nobeforeafter]
#2
\end{tcolorbox}
\end{tcolorbox}
}
\makeatother

\parindent=0pt

%\newcommand{\cornell}[2]

%\AddEverypageHook{
%\hspace{.3\textwidth}\vrule width 3pt depth .4\textheight 
%\vspace{-\textheight}}

\providecommand{\LyX}{L\kern-.1667em\lower.25em\hbox{Y}\kern-.125emX\@}

\begin{document} 
\maketitle
\SetBgContents{\rule[0em]{4pt}{\textheight}}

% Page 1: Extensive vs intensive

\cornell{What is a thermodynamic variable?}{{\bf Thermodynamic variables} \\  Measurable macroscopic quantities that are associated with a macroscopic system.  Generally these are quantities that can be easily measured or controlled in experiments.}
\cornell{What property do extensive variables have?}{{\bf Extensive variables} \\ If the value of a thermodynamic variable depends on the size of the system then it is said to be an extensive quantity.  The values of extensive quantities depend on the number of atoms (mols) that are present in the system.}
\cornell{What property do intensive variables have? \\ \\ Is the density of a extensive variable intensive/extensive? \\ \\ What is a field?}{{\bf Intensive variables} \\ If the value of a thermodynamic variable does not depend on the size of the system then it is said to be an intensive quantity.  The values of intensive quantities do not depend on the number of atoms (mols) that are present in the system.  The density of an extensive variable (e.g. Number of atoms in a volume) is an intensive quantity.  Quantities that are intrinsically intensive (i.e. not densities of extensive quantities) are also called {\bf fields.}  }
\cornell{List as many thermodynamic variables as you can, what symbols are used to represent these quantities and are they intensive or extensive?}{ 
\begin{tabular}{ l | l }
 {\bf Extensive} & {\bf Intensive}  \\ \hline
Volume (V) & Pressure (-P) \\
Strain ($\nu$) & Strain ($\sigma$) \\
Number of atoms of type $i$ ($N_i$) & \emph{Chemical potential of species $i$ ($\mu_i$)} \\
\emph{Entropy (S)} & Temperature (T) \\
Magnetization (M) & Magnetic Field Strength (H) \\
Polarization (P) & Electric Field Strength (E) 
\end{tabular}
}
\cornell[Summary]{}{Thermodynamic variables are measurable macroscopic quantities that can be easily measured in experiments.  They are called extensive variables if their value depends on the number of atoms that are present and intensive if their value is independent of the number of atoms that are present.}

% Page 2 : Equilibrium

\cornell{What does it mean when we state that a system is in an equilibrium state? \\ \\ What happens to the intrinsic and extrinsic variables at equilibrium? \\ \\ How can we characterise the state of the system when it is at equilibrium?}{
{\bf Equilibrium} \\  A system is said to be in equilibrium if it doesn't change over macroscopic timescales.  For example the liquid inside a medical thermometer expands when it is first put into a persons mouth because at that point it is {\bf NOT} in equilibrium.  However, after a short period of time the liquid stops expanding as the thermometer reaches equilibrium with the person (heat bath) with which it has been put into contact.  Once this equilibrium has been reached the volume of the liquid inside the thermometer does not change any further it settles down to a constant value.  At equilibrium the intensive variables are uniform (homogenous) in that that they have the same value for all parts of the system (in our thermometer example the temperature of the liquid in the thermometer and the temperature of the persons mouth are the same).  The extensive variables, meanwhile, do not change in time. {\bf When a system is at equilibrium its ``state" is completely characterised the values of a small set of thermodynamic variables}
 }
\cornell{In general why does the state of a system change?}{{\bf Reversible change} \\ In general, when a system moves from one thermodynamic state to another, it does so because it is no longer in equilibrium.  When this is the case the values of the various thermodynamic quantities change in order to attain a new equilibrium.  For example, when I boil a pan full of water I place the pan in contact with a hot stove (a heat bath).   The system here is the pan of water and the stove.  Obviously, there is no equilibrium here as one of the intensive variables (the temperature) is not homogenous - the stove will be considerably hotter than the water.  Consequently, in this out of equilibrium system the water heats up so as to ensure that the intensive variables (the temperature) take on a uniform, single value across the entire system.  This sort of change in the values of the thermodynamic variables is said to be {\bf irreversible.}  In classical thermodynamics we prefer, for reasons of mathematical convenience, to think about reversible changes.  I have never really found a satisfactory and simple way of explaining what distinguishes reversible transitions from irreversible ones.  Oftentimes what is written in textbooks is either so vague as to be meaningless (for example in some places you read that reversible transitions take place infinitely slowly) or clearly wrong as it is straightforward to think of processes that are reversible that do not have the properties ascribed to reversible transitions in the book.   The essential point though is that when the transition takes place reversibly all the various processes that absorb or release energy are described mathematically in the model.  When the transition takes place irreversibly there are hidden processes (such as friction) that absorb/release energy.  As there is no mathematical description of these processes appears in the model it thus appears as if the model does not conserve energy.  Obviously, energy must be convserved in reality, however, so this distinction between reversible and irreversible transitions must be a feature of the model.  In reality all thermodynamic transitions are reversible.}

\cornell[Summary]{}{A system is said to be at equilibrium if it does not change over macroscopic timescales.  At equilibrium extensive variables do not change in time and intensive variables take on a single, uniform value across the whole system.  Equilibrium states are completely characterised by a small set of thermodynamic variables.}

% Page 3 : Functions of state and Gibb's phase rule

\cornell{What is Gibbs phase rule? \\ \\ What do we mean by the number of components, $C$? \\ \\ What do we mean by the number of phases, $\pi$? \\ \\ How many independent thermodynamic variables are there for a one phase system containing one chemical component? \\ \\ What is a function of state? }{{\bf Equilibrium and Gibb's phase rule} \\ The requirement that all intensive variables must be homogenous (equal) across the entirety of the system when the system is at equilibrium ensures that {\bf at equilibrium} not all thermodynamic quantities are independent.  In particular, there is a result called Gibb's phase rule which states:
\begin{equation}
F = C - \pi + 2
\end{equation}
Here $C$ is the number of components in the system\footnote{the number of chemically distinct species}, $\pi$ is the number of phases in the system\footnote{the number of distinct regions of space in the system where the properties of the material are essentially uniform} and $F$ is the number of thermodynamic variables that are independent.  Consider the consequence this has for argon gas.  Argon gas contains only one chemical component (argon atoms) hence $C=1$.  Furthermore,  the properties of the gas are uniform throughout so the number of phases in this system is also equal to 1 ($\pi=1$).  The thermodynamic state of argon gas ({\bf if it is at equilibrium}) can thus be characterised using only 2 thermodynamic variables. If we are told the temperature, $T$, of the gas and the pressure, $P$, that it is under there should be some function, $f$, that we can use to calculate the volume, $V$, of the gas, $V = f(P,T)$.
Now consider some argon gas placed next to some water.  Some of the argon atoms will dissolve in the water so there are 3 chemically distinct species\footnote{argon atoms in the gas, argon atoms dissolved in the water and water molecules} so $C=3$.  There are also two distinct phases\footnote{the argon solution and the argon gas} so $\pi=2$.  The number of variables $F$ must thus be $F=C-\pi+2=3-2+2 = 3$.  The equilibrium thermodynamic state of the system can thus be described using three variables - for instance the temperature, the pressure and the concentration of argon atoms in the solution.  All other quantities - e.g. the volume of the whole system, the chemical potential - must be some function of these three variables. 
   }

\cornell{Write out the ideal gas equation? \\ \\ Write out a function of state for a real gas?  \\ \\ Why is number of mols not a thermodynamic variable?}{{\bf Ideal and Non-Ideal gasses} \\ For gasses it is possible to write out explicit forms for the equations of state.  In the first instance these relations were obtained through experimentation - i.e. looking at what happens to the volume of the gas as the pressure and temperature are changed and fitting.  We thus have:
\begin{equation}
PV = nRT
\end{equation}
where $n$ is the number of mols of gas\footnote{The number of mols, $n$, in the system is not a thermodynamic variable as it is a fixed property of the system.  When the number of atoms/mols appears as a thermodynamic variable there must be at least two phases present.} and $R$ is a fixed constant that is the same for all gases.  This is the so called ideal gas relation (we will see why in future lectures).  The behaviour of many real gasses is better described using the following equation:
\begin{equation}
\left( P + \frac{a}{V^2} \right)(V-b) = nRT
\end{equation}
where $a$ and $b$ are constants that depend on the particular type of gas.   
}

\cornell[Summary]{}{The number of independent thermodynamic variables is fixed at equilibrium.  Consequently, for a system {\bf at equilibrium} we can write functions that relate the values of the various thermodynamic variables.  }


% Page 4 : Work

\cornell{How do you calculate the work done when an object is moved from $x$ to $x+\Delta x$ against a force $F(x)$?}{{\bf Work in Newtonian physics} \\ To lift a heavy object you have to expend some effort.  By contrast when you drop it you expend zero effort - it will accelerate towards the earth because it is acted upon by the force of gravity.  Early on in your scientific education you will have learnt that when lifting an object you have to do work on it against the force of gravity or that you have to give the box more potential energy.  If you think about the meaning of words in this statement what you are saying it is saying that you are giving the box the potential to accelerate down to the earth at some point in the future i.e. when it is released.  When you were told this you should have been told that the work done, $\Delta w$, in lifting the box from point $x$ to point $x + \Delta x$ is given by:
\begin{equation}
\Delta w = \int_{x}^{x + \Delta x} \mathbf{F}(x') \textrm{d}x'
\nonumber
\end{equation}
where $\mathbf{F}(x')$ is a {\bf vector} quantity - the force - that acts on the box when it is at point $x'$.  Probably when you first encountered this equation in the context of box lifting you would have been taught that $\Delta w = mg \Delta x$ but as you are all now grown ups you will realise that this is just the above integral with $\mathbf{F}(x')$ equal to the constant function $mg$.  
    }

\cornell{What work is done by a gas at pressure, P, when it changes its volume (reversibly) from $v$ to $v+\Delta v$?  \\ \\ Why is only possible to use this expression if the change in volume is done reversibly?  \\ \\ Why can the pressure be calculated given the values of the volume and the temperature? }{{\bf Expansion of gasses} \\ Suppose that I have a piston with a heavy head that contains some gas as shown in the figure.  For the gas to expand it must do some work in order to lift up the piston head.  This work comes about because the gas exerts a force on the piston head.  Furthermore, we know that the pressure, $P$, is defined as the force $\mathbf{F}$ over the area of the piston head, $A$ - i.e. as $P=\frac{F}{A}$.  We can thus write that the work done, $\Delta w$, when the piston head moves from point $x$ to point $x+\Delta x$ is given by:
\begin{equation}
\Delta w = -\int_{x}^{x+\Delta x} \mathbf{F}(x) \textrm{d}x' = = -\int_{x}^{x+\Delta x} \frac{\mathbf{F}(x)}{A} A \textrm{d}x' -\int_{v}^{v+\delta v} P(V,T) \textrm{d}V
\end{equation}
The third equality here follows because $A\textrm{d}x'$ - the cross sectional area multiplied by the change in the position of the piston head - is equal to the change in the volume of the gas, $\textrm{d}V$.  Notice furthermore that {\bf this equation only holds when the the change in gas's volume takes place reversibly.}  This means that in our model we are assuming that the intensive quantity - the pressure - is distributed uniformly throughout the system.  If this were not the case the change would take place irreversibly as processes would be occurring for which we have no mathematical description in the model.  A key point here is that, because the transition is reversible, the state of the system can be characterised using a small set of thermodynamic variables at all points during the transition.  If the transition were taking place irreversibly the pressure would be poorly defined.   We thus would be unable to calculate its value using the function of state, $P(V,T)$ as we have in the above integral. 
}

\cornell[Summary]{}{Work done can be calculated using $\Delta w = \int_{x}^{x + \Delta x} \mathbf{F}(x') \textrm{d}x'$}

% Generalised forces

\cornell{List all the extensive variables you know together with their corresponding, conjugate intensive variables? \\ \\ How do you calculate the work done, $\Delta w_{rev}$ when an extensive variable reversibly changes it's value from $e$ to $e+\Delta e$? \\ \\ Why can we only calculate the change in work like quantity when the state of the system changes reversibly? \\ \\ Are intensive variables scalar or vector quantities? \\ \\ Are extensive variables scalar or vector quantities? }{{\bf Intensive quantities as generalised forces} \\ Classical thermodynamics is a phenomenological theory that works by exploiting an analogy with Newtonian physics.  Extensive quantities are thought of as generalised coordinates, while intensive quantities are thought of as ``forces" that act on these generalised coordinates.  We thus write the work done in changing any general extensive quantity, $E$, from $e$ to $e+\Delta e$ as: 
\begin{equation}
\Delta w = \int_e^{e+\Delta e} \mathbf{I}(E,\{\Gamma\}) \textrm{d}E
\label{eqn:gen}
\end{equation}
where $\mathbf{I}(E,\{\Gamma\})$ is the value taken by the intensive quantity, $\mathbf{I}$ when the extensive quantity has a value equal to $E$ and the other thermodynamic variables that characterise the system have the set of values $\{\Gamma\}$ - notice that we are once again calculating the values of the intensive function using a function of state.  There are a number of important things to note with regards to this expression:
\begin{itemize}
\item Every extensive quantity has a corresponding (conjugate) intensive quantity and these two variables appear together when calculating the change in the work-like quantity using the integral in equation \ref{eqn:gen} .  For some intensive quantities, e.g. pressure it is immediately obvious what the conjugate variable should be - volume.  By contrast, for intensive quantities such as temperature or extensive quantities such as number of atoms what the conjugate variable should be is far from obvious.  In fact we have to introduce new extensive variables in these cases - entropy and the chemical potential.  The table on page 1 of these notes shows which pairs of extensive and intensive variables are conjugate.
\item Intensive quantities should be thought of as vectors as they have both magnitude and direction.  We will see in subsequent lectures that intensive quantities are derivatives and you should by now know that the derivative of a scalar-valued function is always a vector.  In thermodynamics the sign of the intensive quantity tells us what is doing the work and what is being worked upon.  So, for example, a pressure is a force applied by the external world to a gas.  As such when a gas expands it does work against this force.       In other words the system (the gas) does work on the rest of the world.
\item {\bf Equation \ref{eqn:gen} is only valid when the extensive variable's value is changed reversibly.}  This is hugely important - when we use this expression we are assuming that our mathematical model provides a complete description of the phenomenon under study. 
\item When equation \ref{eqn:gen} is used to calculate the work done during a transition the only thermodynamic variables that change during the transition are the extensive variable, $E$, and its conjugate intensive variable.  All the other thermodynamic variables, $\{\Gamma\}$, are kept fixed during the transition so that the value of the intensive variable $\mathbf{I}$ can be calculated using the function of state.    
\end{itemize}
 }

\cornell[Summary]{}{The work done when an extensive variable, $E$, changes from $e$ to $e+\Delta e$ is given by $\textrm{d}w_{rev} = \int_e^{e+\Delta e} \mathbf{I}(E,\{\Gamma\}) \textrm{d}E$, where $\mathbf{I}(E)$ is the value of the conjugate intensive quantity and $\{\Gamma\}$ are the fixed values of the set of thermodynamic variables that are required to define the state of the system - the number of variables in the set will be determined by the Gibbs phase rule.} 

\cornell{What do the following terms mean: isothermal, isobaric, isochoric and adiabatic?}{{\bf Changes of states with constraints} \\  We have many words to describe what constraints have been put in place during a (generally reversible) transition.
\begin{itemize}
\item {\bf Isothermal} - \emph{temperature} is kept fixed during the transition
\item {\bf Isobaric} - \emph{pressure} is kept fixed during the transition
\item {\bf Isochoric} - \emph{volume} is kept fixed during the transition
\item {\bf Isoentropic} or {\bf Adiabatic} - \emph{entropy} is kept fixed during the transition
\end{itemize}
    }
    
\cornell{What characterises adiabatic and diathermal walls?  What can be transferred through these walls?}{{\bf Walls} \\ In thermodynamics it is critical to define how the system under study interacts with the rest of the universe.  In actuality all systems will exchange heat and matter with the external world to a certain degree.  In physics, however, we consider a simplified version of reality when we construct our models and theories.  We thus construct barriers (walls) between the system and the rest of the universe if you will of the following types:
\begin{itemize}
\item {\bf adiabatic} - a wall that \emph{does not} allow \emph{heat\footnote{The change in heat, $\Delta q_{rev}$, that takes place during a reversible transition is given by $\Delta q_{rev} = \int_s^{s+\Delta s} T(S,V)\textrm{d}S$.  Heat is thus a work-like quantity where the extensive ``coordinate" is the (at this stage rather mysterious) entropy, $S$, and the intensive ``force" is temperature $T$.} or matter} to cross it.  The system \emph{can exchange} mechanical \emph{work} with the outside world through the wall.  A system with adiabatic walls has an entropy that is fixed.
\item {\bf diabatic/diathermal} - a wall that \emph{does not} allow the system to do \emph{work} on the universe and that \emph{does not} permit \emph{exchange of matter} with the universe.  However, the system \emph{can exchange heat} with the universe through the wall.   A system with diabatic walls has a fixed volume. 
\end{itemize}
   }  
   
\cornell{What is an isolated system? \\ \\ What is a closed system? \\ \\ What is an open system?}{{\bf Closed, Open and Isolated Systems} \\ A last important piece of nomenclature concerns the type of system.  In particular we have the following three types:
\begin{itemize}
\item {\bf Isolated} - systems that cannot do work on the rest of the universe or exchange heat with the rest of the universe.  These systems also cannot exchange matter with the rest of the universe - the number of atoms (mols) contained within them is fixed.
\item {\bf Closed} - systems that can do work on the rest of the universe and that can exchange heat with the rest of the universe.  However, these systems cannot exchange matter with the rest of the universe - the number of atoms (mols) contained within them is fixed. Most of the closed systems you will encounter in this course are one phase systems with one chemical component.  Consequently, the Gibbs phase rule tells us that their thermodynamic state can be specified using only 2 thermodynamic variables.  
\item {\bf Open} - systems that can do work on the rest of the universe and that can exchange heat with the rest of the universe.  These systems are also allowed to exchange matter with the rest of the universe - they do not contain a fixed number of atoms. 
\end{itemize}
}  

\end{document}