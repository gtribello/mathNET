\documentclass[a4paper]{article} 
\usepackage{tcolorbox}
\usepackage{amsmath,amsfonts,amsthm}
\tcbuselibrary{skins}
\usepackage{graphicx}
\usepackage{hyperref}

\makeatletter
\def\cornell{\@ifnextchar[{\@with}{\@without}}
\def\@with[#1]#2#3{
\begin{tcolorbox}[enhanced,colback=white!15,colframe=white,colupper=gray]
\begin{tcolorbox}[enhanced,colback=gray,colframe=black,fonttitle=\large\bfseries\sffamily,sidebyside, nobeforeafter,colupper=black,
righthand width=14cm,
opacityframe=0,opacityback=0.3,opacitybacktitle=1, opacitytext=1,
segmentation style={black!55,solid,opacity=0,line width=1pt},
title=#1
]
%\begin{tcolorbox}[colback=red!05,colframe=red!25,sidebyside align=top,
%width=\textwidth,nobeforeafter]#2\end{tcolorbox}%
%\tcblower
%\sffamily
%\begin{tcolorbox}[colback=blue!05,width=\textwidth]
% #3
%\end{tcolorbox}
\Huge {\bf  #2}
\tcblower
#3
\end{tcolorbox}
\end{tcolorbox}
}
\makeatother

\title{
\vspace{-3em}
\begin{tcolorbox}
\Huge\sffamily AMA4004 Statistical mechanics: Ensembles  
\end{tcolorbox}
\vspace{-3em}
}

\date{}

%\usepackage{background}
%\SetBgScale{1}
%\SetBgAngle{0}
%\SetBgColor{red}
%\SetBgContents{\rule[0em]{4pt}{\textheight}}
%\SetBgHshift{-2.3cm}
%\SetBgVshift{0cm}
\usepackage{lipsum}% just to generate filler text for the example
\usepackage[margin=1.5cm]{geometry}
\usepackage{lipsum}% just to generate dummy text for the example


\begin{document}
\maketitle

{\bf This assignment contains no hard parts and as such if this report is handed in for the portfolio the maximum mark you can get for it is 8/12.}

In order to do this exercise you will need to work through the following materials:

\begin{itemize}
\item \href{http://gtribello.github.io/mathNET/GENERALIZED\_PARTITION\_FUNCTION.html}{http://gtribello.github.io/mathNET/GENERALIZED\_PARTITION\_FUNCTION.html}
\item \href{http://gtribello.github.io/mathNET/CANONICAL\_ENSEMBLE.html}{http://gtribello.github.io/mathNET/CANONICAL\_ENSEMBLE.html}
\item \href{http://gtribello.github.io/mathNET/ISOTHERMAL\_ISOBARIC\_ENSEMBLE.html}{http://gtribello.github.io/mathNET/ISOTHERMAL\_ISOBARIC\_ENSEMBLE.html}
\end{itemize}

You must prepare a short report (no more than 3 pages) on one of the following six ensembles.  (I will tell you which you are to do)

\begin{itemize}
\item The canonical (NVT) ensmeble
\item The canonical (NHT) ensemble - H being magnetic field strength
\item The Isothermal-Isobaric (NPT) ensemble 
\item The grand canonical ($\mu$VT) ensemble
\item The grand canonical ($\mu$PT) ensemble
\item The microcanonical ensemble
\end{itemize}

If your report is on any of the first five of the ensembles above you should:

\begin{itemize}
\item Write about which thermodynamic variables are constrained in this particular ensemble.
\item Derive an expression for the probability of being in a microstate in this ensemble.
\item Derive an expression that can be used to calculate the partition function.
\item Show that certain ensemble averages can be calculated by taking suitable derivatives of the logarithm of the partition function.
\item Explain how you can relate the logarithm of the partition function of this ensemble to a particular thermodynamic potential.
\item Derive an expression that relates fluctuations in the value of a extensive quantity with a response function. 
\end{itemize}

If you report is on the microcanonical ensemble you should:

\begin{itemize}
\item Explain what the probability of being in a particular state is equal to for the microcanonical ensemble.
\item Give an expression for the microcanonical partition function.
\item Show how partition functions for all other ensembles can be calculated from the microcanonical partition function.
\item Discuss how partition functions that incorporate constraints on microscopic coordinates can thus be calculated.
\item Explain the meaning of the term ``Potential of mean force"
\end{itemize}

\end{document}
