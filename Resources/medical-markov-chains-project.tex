\documentclass[paper=a4, fontsize=11pt]{scrartcl}
\usepackage[T1]{fontenc}
%\usepackage{fourier}

\usepackage[english]{babel}
\usepackage[protrusion=true,expansion=true]{microtype}	
\usepackage{amsmath,amsfonts,amsthm} % Math packages
\usepackage[pdftex]{graphicx}	
\usepackage{url}
\usepackage{makecell,pict2e}
\usepackage{comment}
\usepackage{color}

\renewcommand\theadfont{\large}
\newcommand{\gt}[1]{{\color{blue}#1}}
\newcommand{\red}[1]{{\color{red}#1}}

\renewcommand\theadfont{\large}

%%% Custom sectioning
\usepackage{sectsty}
\allsectionsfont{\normalfont\scshape}

\includecomment{answers}
% \excludecomment{answers}

\usepackage[margin=2cm]{geometry}
\setlength{\topmargin}{-2.cm}
\setlength{\headheight}{1cm}

%%% Custom headers/footers (fancyhdr package)
\usepackage{fancyhdr}
\pagestyle{fancyplain}
\fancyhead{}
\fancyfoot[L]{}
\fancyfoot[C]{}
\fancyfoot[R]{\thepage}
\renewcommand{\headrulewidth}{0pt}
\renewcommand{\footrulewidth}{0pt}
\setlength{\headheight}{13.6pt}

%%% Equation and float numbering
\numberwithin{equation}{section}
\numberwithin{figure}{section}
\numberwithin{table}{section}


%%% Maketitle metadata
\newcommand{\horrule}[1]{\rule{\linewidth}{#1}}
\newcommand{\vek}[1]{\mbox{\boldmath $  #1$}}
\newcommand{\ex}[1]{\ensuremath {\mathbb{E}} \left[ #1 \right]}
\newcommand{\var}[1]{\ensuremath{{\rm var}\left[ #1 \right]}}

\title{\usefont{OT1}{bch}{b}{n} \normalfont \normalsize \textsc{SOR3012:
Stochastic Processes} \\ [25pt] \horrule{0.5pt} \\[0.4cm] 
\huge Medical Markov Chains Project \\
\horrule{2pt} \\[0.25cm]
}
\author{ \normalfont
\normalsize
        Gareth Tribello \\[-3pt] \normalsize
        \today
}
\date{}

\begin{document}
\maketitle

For this project you must produce a {\bf three page} set of notes on how the progression of diseases can be modelled using continuous time Markov chains.  You should prepare your report as an 
ipython notebook and within it you should present:

\begin{itemize}
 \item You should explain how Markov chains can be used to describe diseases that progress through a series of states. You should include transition graphs, jump rate matrices and suitable 
 differential equations for these models.
 
 \item You should be explain how Markov chains can be used to describe diseases that progress at different rates and through different stages in different patients.  Again you should include transition graphs, jump rate matrices and suitbale differential equations for these models.

 \item You should describe how we model the situation when diseases like those described in the previous two examples can be reversed.  Once again transition graphs, jump rate matrices and suitable mathematical derivations should be included in your report on these models.

\item You should discuss how numerical simulations of these various models can be performed and on should perform some tests to confirm that your numerical simulations reproduce what is predicted by the analytical models.

 \item You should discuss what limitations there are in using Markov models to describe the progression of diseases and you should provide information on ways that these models can be improved by relaxing the Markov approximation.
\end{itemize}

You should also explain any results you have used on series expansions or special integrals within your notes and provide the reader with links explaining where they can find material that explains 
these ideas.  These links might be pointers to particular websites, pointers to notes from modules that you have studied at Queen's or pointers to websites.  You {\bf should not} get bogged down in 
explaining what a Markov chain is, how the Kolmogorov equations are derived and so on. You can assume that the readers of your report will have studied what you have in SOR3012 and that they will thus
be familiar with these details.

Your report should be output as a pdf from the python notebook software and handed in through the online submission system.

You are strongly encouraged to discuss your report with your colleages.  One thing you might consider doing is using the markscheme below to mark each others report.  You can then give each other 
constructive feedback on what has been done well and what has been done poorly.

On my website you will find an example report on the exponential random variable so that you have some guidance as to what I expect.

\section{Markscheme}

\begin{center}
\begin{tabular}{ l | c | c }
Description & Marks available & Final mark \\ \hline
Discussion of how states and chains describe progression of disease & 2 & \\
Transition graphs & 1 & \\
Jump rate matrices & 1 & \\
Classifications of state & 1 & \\
Discussion of chains with stationary/no stationary distribution & 1 & \\
Derivations using Kolmogorov forward equations & 2 & \\
Numerical simulations & 2 & \\
Comparison of numerical and analytical results & 2 & \\
Discussion of limitations of Markov assumption and extensions & 2 & \\
References were stated & 1 
\end{tabular}
\end{center}


\end{document}
