\documentclass[paper=a4, fontsize=11pt]{scrartcl}
\usepackage[T1]{fontenc}
%\usepackage{fourier}

\usepackage[english]{babel}
\usepackage[protrusion=true,expansion=true]{microtype}	
\usepackage{amsmath,amsfonts,amsthm} % Math packages
\usepackage[pdftex]{graphicx}	
\usepackage{url}
\usepackage{makecell,pict2e}
\usepackage{comment}
\usepackage{color}
\usepackage{hyperref}

\renewcommand\theadfont{\large}
\newcommand{\gt}[1]{{\color{blue}#1}}
\newcommand{\red}[1]{{\color{red}#1}}

\renewcommand\theadfont{\large}

%%% Custom sectioning
\usepackage{sectsty}
\allsectionsfont{\normalfont\scshape}

\includecomment{answers}
% \excludecomment{answers}

\usepackage[margin=2cm]{geometry}
\setlength{\topmargin}{-2.cm}
\setlength{\headheight}{1cm}

%%% Custom headers/footers (fancyhdr package)
\usepackage{fancyhdr}
\pagestyle{fancyplain}
\fancyhead{}
\fancyfoot[L]{}
\fancyfoot[C]{}
\fancyfoot[R]{\thepage}
\renewcommand{\headrulewidth}{0pt}
\renewcommand{\footrulewidth}{0pt}
\setlength{\headheight}{13.6pt}

%%% Equation and float numbering
\numberwithin{equation}{section}
\numberwithin{figure}{section}
\numberwithin{table}{section}


%%% Maketitle metadata
\newcommand{\horrule}[1]{\rule{\linewidth}{#1}}
\newcommand{\vek}[1]{\mbox{\boldmath $  #1$}}
\newcommand{\ex}[1]{\ensuremath {\mathbb{E}} \left[ #1 \right]}
\newcommand{\var}[1]{\ensuremath{{\rm var}\left[ #1 \right]}}

\title{\usefont{OT1}{bch}{b}{n} \normalfont \normalsize \textsc{SOR3012:
Stochastic Processes} \\ [25pt] \horrule{0.5pt} \\[0.4cm] 
\huge Programming a random walk \\
\horrule{2pt} \\[0.25cm]
}
\author{ \normalfont
\normalsize
        Gareth Tribello \\[-3pt] \normalsize
        \today
}
\date{}

\begin{document}
\maketitle

Write a python notebook that contains a program that will generate a simple random walk in one dimension with adsorbing boundary conditions.  
By running the program in your python notebook multiple times calculate estimates for the following:

\begin{enumerate}
\item The probability that the walk finishes at $X=0$ together with error bars at the 90 \% confidence limit on this estimate.
\item The probability mass function for the number of steps in the walk prior to absorbtion.
\end{enumerate}

Compare the results you obtain with the values obtained by inserting the parameters of your model into the expression that can be by solving the 
relevant homogeneous and inhomogeneous difference equation.  Do not include derivations of these analytical result in your report, however. 

\end{document}
