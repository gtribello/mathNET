\documentclass[a4paper]{article}

\usepackage[margin=2.5cm,headheight=50pt,includeheadfoot]{geometry}

\usepackage{amsfonts}
\usepackage{amsmath}
\usepackage{graphicx}
\usepackage{fancyhdr}
\pagestyle{fancy}
\renewcommand{\headrulewidth}{2pt}

\usepackage{xcolor}

\rhead{\includegraphics[width=5cm]{../../html/assets/img/logo.png}}
\lhead{\Huge SOR3012: Tutorial week 5}

\begin{document}

\section{Aim}

The aim of the tutorial this week is to look at the feedback on your random variable reports and to think about what to do differently the next time you do a report like this one.

\section{Bring}

Please bring multiple copies of the report on random variables that you handed in to be marked as well as multiple copies of the sheet with the feedback that I will have returned to you by this 
stage.  In addition, bring the report you wrote on the work that you did last week that you should have handed in on Tuesday.  In this report you should have discussed the feedback that you got from 
me on the project that you have written.

\section{Approach}

You will work in groups of five.  Each of you should read the report that was written by each your colleagues again as well as the comments that I made on their report. You should then read what your 
colleague has written about how they would do things differently the next time they prepare a report given what I said in my feedback.  Once you have read all this material write some 
feedback on one of your colleagues reports and their plans for the next report.  Your feedback should consist of at least three sentences of continuous prose.  In addition, as you write feedback you 
should reflect on your own report as you read those of your colleagues and try to determine if there is anything that they have done that you should also do.  

You will be given copies of the report descriptions for each of the five random variable projects again.  Your can again use the mark scheme in the project description for reference although it is 
necessary to determine which of the marks you would have awarded as by now I will have awarded marks and my marks are definitive.  In the feedback you write on your colleagues reports you may choose 
to address the following questions:

\begin{enumerate}
 \item Which items in the mark scheme were not awarded?  Why were these marks awarded?  In other words, what was missing from the report? 
 \item What did the comments on the report say about clarity of prose and report structure?  What single thing should definitely be changed prior to preparing the next report? 
 \item Where did you do better than the person whose report you are reading?  Is there any critical feedback that you can give your colleague on how to improve these areas?
 \item Where did you do worse than the person whose report you are reading?  Is there anything where they can perhaps give you advice on how to improve your future reports?
\end{enumerate}

You do not need to write your piece in silence and you can (and should) discuss things with your colleagues before you write them down.  I do, however, have an expectation that you will be able to 
write something about each of the four colleagues in your group during the session and I will be going around to check what you are writing.



\end{document}
