\documentclass[paper=a4, fontsize=11pt]{scrartcl}
\usepackage[T1]{fontenc}
%\usepackage{fourier}

\usepackage[english]{babel}
\usepackage[protrusion=true,expansion=true]{microtype}	
\usepackage{amsmath,amsfonts,amsthm} % Math packages
\usepackage[pdftex]{graphicx}	
\usepackage{url}
\usepackage{makecell,pict2e}
\usepackage{comment}
\usepackage{color}

\renewcommand\theadfont{\large}
\newcommand{\gt}[1]{{\color{blue}#1}}
\newcommand{\red}[1]{{\color{red}#1}}

\renewcommand\theadfont{\large}

%%% Custom sectioning
\usepackage{sectsty}
\allsectionsfont{\normalfont\scshape}

\includecomment{answers}
% \excludecomment{answers}

\usepackage[margin=2cm]{geometry}
\setlength{\topmargin}{-2.cm}
\setlength{\headheight}{1cm}

%%% Custom headers/footers (fancyhdr package)
\usepackage{fancyhdr}
\pagestyle{fancyplain}
\fancyhead{}
\fancyfoot[L]{}
\fancyfoot[C]{}
\fancyfoot[R]{\thepage}
\renewcommand{\headrulewidth}{0pt}
\renewcommand{\footrulewidth}{0pt}
\setlength{\headheight}{13.6pt}

%%% Equation and float numbering
\numberwithin{equation}{section}
\numberwithin{figure}{section}
\numberwithin{table}{section}


%%% Maketitle metadata
\newcommand{\horrule}[1]{\rule{\linewidth}{#1}}
\newcommand{\vek}[1]{\mbox{\boldmath $  #1$}}
\newcommand{\ex}[1]{\ensuremath {\mathbb{E}} \left[ #1 \right]}
\newcommand{\var}[1]{\ensuremath{{\rm var}\left[ #1 \right]}}

\title{\usefont{OT1}{bch}{b}{n} \normalfont \normalsize \textsc{SOR3012:
Stochastic Processes} \\ [25pt] \horrule{0.5pt} \\[0.4cm] 
\huge Normal random variable project \\
\horrule{2pt} \\[0.25cm]
}
\author{ \normalfont
\normalsize
        Gareth Tribello \\[-3pt] \normalsize
        \today
}
\date{}

\begin{document}
\maketitle

For this project you must produce a {\bf three page} set of notes on normal random variables.  You should prepare your report as an ipython notebook and within it you should present:

\begin{itemize}
 \item An explanation on the type of experiments that this random variable can be used to model. 
 \item A statement of the probablity density function for the random variable.
 \item A proof that the random variable is properly normalized.
 \item A statement of the expectation of the random variable and a proof that this is indeed the expectation.
 \item A stement of the variance of the random variable and a proof that this is indeed the variance.
 \item A discussion of how random variables of this type can be generated by a computer program and a python function that generates normal random variables.
 \item A discussion of how a sample of random variables of this type can be generated by a computer and a python function that generates a sample of binomial random variables, the median of this sample and the 25 and 75th percentiles of the sample.  Please include the code that you used to calculate these quantities.
 \item A graph showing the sample mean for this random variable calculated from different numbers of independent samples.  Please include the code that you used to generate this plot.
 \item A graph showing the variance for this random variable calculated from differnet numbers of independent samples.  Please include the code that you used to generate this plot.
 \item A graph showing an estimated histogram that you calculated by generating multiple uniform random variables.  Again please include the code that you used to generate this plot.
 \item A short list of references that you used to compile the report.
\end{itemize}

You should also explain any results you have used on series expansions or special integrals within your notes and provide the reader with links explaining where they can find material that explains 
these ideas.  These links might be pointers to particular websites, pointers to notes from modules that you have studied at Queen's or pointers to websites.

You should not get bogged down in the abstract details of probablity theory, the way expectation is calculated and so on.  You can assume that the readers of your report will know these details.

Your report should be output as a pdf from the python notebook software and handed in through the online submission system.

You are strongly encouraged to discuss your report with your colleages.  One thing you might concisder doing is using the markscheme below to mark each others report.  You can then give each other 
constructive feedback on what has been done well and what has been done poorly.

On my website you will find an example report on the Bernoulli random variable so that you have some guidance as to what I expect.

\section{Markscheme}

\begin{center}
\begin{tabular}{ l | c | c }
Description & Marks available & Final mark \\ \hline
All series expansion/integral results were explained & 1 &  \\
Resources for series expansions/integrals were identified & 1 & \\
Probability mass/density function was stated & 1 & \\
Derivation of mass/density function was explained & 1 & \\
Probability mass/density function was shown to be normalised & 1 & \\
Expectation was stated & 1 & \\
Proof of expectation was provided & 1 & \\
Variance was stated & 1 & \\
Proof of variance was provided & 1 &  \\
Code for generating random variables was provided & 1 & \\
Median value was computed & 1 & \\ 
Percentiles were computed & 2 & \\ 
Plot of sample mean was provided & 2 & \\
Plot of variance as function of sample size was provided & 2 & \\
Plot of histogram was provided & 2 & \\
References on random variable were stated & 1 &
\end{tabular}
\end{center}

\end{document}
