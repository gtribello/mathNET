\documentclass[a4paper]{article} 
\usepackage{tcolorbox}
\usepackage{amsmath,amsfonts,amsthm}
\tcbuselibrary{skins}
\usepackage{graphicx}
\usepackage{hyperref}

\makeatletter
\def\cornell{\@ifnextchar[{\@with}{\@without}}
\def\@with[#1]#2#3{
\begin{tcolorbox}[enhanced,colback=white!15,colframe=white,colupper=gray]
\begin{tcolorbox}[enhanced,colback=gray,colframe=black,fonttitle=\large\bfseries\sffamily,sidebyside, nobeforeafter,colupper=black,
righthand width=14cm,
opacityframe=0,opacityback=0.3,opacitybacktitle=1, opacitytext=1,
segmentation style={black!55,solid,opacity=0,line width=1pt},
title=#1
]
%\begin{tcolorbox}[colback=red!05,colframe=red!25,sidebyside align=top,
%width=\textwidth,nobeforeafter]#2\end{tcolorbox}%
%\tcblower
%\sffamily
%\begin{tcolorbox}[colback=blue!05,width=\textwidth]
% #3
%\end{tcolorbox}
\Huge {\bf  #2}
\tcblower
#3
\end{tcolorbox}
\end{tcolorbox}
}
\makeatother

\title{
\vspace{-3em}
\begin{tcolorbox}
\Huge\sffamily AMA4004 Statistical mechanics: Equipartition  
\end{tcolorbox}
\vspace{-3em}
}

\date{}

%\usepackage{background}
%\SetBgScale{1}
%\SetBgAngle{0}
%\SetBgColor{red}
%\SetBgContents{\rule[0em]{4pt}{\textheight}}
%\SetBgHshift{-2.3cm}
%\SetBgVshift{0cm}
\usepackage{lipsum}% just to generate filler text for the example
\usepackage[margin=1.5cm]{geometry}
\usepackage{lipsum}% just to generate dummy text for the example


\begin{document}
\maketitle

In order to do this exercise you will need to work through the following topics:

\begin{itemize}
\item \href{http://gtribello.github.io/mathNET/IDEAL\_GAS.html}{http://gtribello.github.io/mathNET/IDEAL\_GAS.html}
\item \href{http://gtribello.github.io/mathNET/PARTITION\_FUNCTION\_FOR\_MOLECULES.html}{http://gtribello.github.io/mathNET/PARTITION\_FUNCTION\_FOR\_MOLECULES.html}
\end{itemize}

\begin{enumerate}

\item Discuss how the partition function for a molecule can be calculated by decomposing the partition function into vibrational, translational and rotational parts.  In your discussions you should make sure that you explain clearly what assumptions are made in this approach and cases where those assumptions might be invalid.

\item Consider a single particle constrained in one-dimensional box of length $L$.  Use what you know about ideal gasses to show that the canonical partition function for this system is given by:
$$
Z = \frac{L}{h} \left( \frac{2\pi m}{\beta} \right)^\frac{1}{2} 
$$
Use this expression to determine expressions for the average energy of the particle as a function of temperature and the heat capacity for this system as a function of temperature.

\item {\bf (Hard)} Consider a diatomic molecule whose potential energy as a function of the distance, $r$, between the two atoms of which it is composed is given by some complicated function $V(r^2)$.  Explain, using the Taylor series and what you have learnt about harmonic oscillators, why such potentials are commonly written as a sum of a harmonic and an anharmonic part.  Which, typically, is the larger of the two terms in this expression?  If one neglects the smaller of these two terms when calculating the energy what features in the energy landscape for the chemical bond in the diatomic are missing?

\item Give an expression for the energy levels of a one dimensional harmonic oscillator and use this expression to show that the canonical partition function for such systems is given by:
$$
Z  = \frac{ e^{-\frac{\beta \hbar \omega}{2}} }{ 1 - e^{-\beta \hbar \omega} }
$$
Use this result to determine expressions for the average energy of the particle as a function of temperature and the heat capacity for this system as a function of temperature.  Plot a graph of the heat capacity as a function of temperature for this system together with a graph showing the behaviour of heat capacity as a function of temperature for the system discussed in question 2.  Discuss what value you observe for the heat capacity at high temperature in these two systems and why this behaviour is in accordance with the equipartition principle.

\item Discuss the behaviour of the heat capacity of the harmonic oscillator and the particle in a box in the low temperature limit.  In which of these two systems does the behaviour of the heat capacity deviate from the predictions of equipartition?  Why does this deviation occur for this particular system and why does it not occur for the other system you studied?  

\item Lets now consider a system of $N$ uncoupled harmonic oscillators with characteristic frequencies $\{ \omega_1, \omega_2, \dots \omega_N\}$.  Explain what the heat capacity for this system will equal in the high temperature limit {\bf without} doing any derivations.

\item {\bf (Hard)} Explain why the canonical partition function for this system of $N$ uncoupled harmonic oscillators is given by:
$$
Z_c = \frac{e^{-\beta E_0}}{2^N \prod_{i=1}^N \sinh\left( \frac{\beta \hbar \omega_i}{2} \right)}
$$      
where $E_0$ is the energy of the ground state of the system.  Hence show that the behaviour of the heat capacity in the high temperature limit is in accordance with what you predicted in part 6.

\item {\bf (Hard)} Consider a system of $N$ coupled harmonic oscillators.  Explain what value the heat capacity for this system must take in the high temperature limit.  Discuss why the partition function for this system has the same functional form as the partition function for the uncoupled oscillators discussed in part 7 and explain how the values of $\omega$ are determined from the dynamical matrix in the coupled case.


\end{enumerate}



\end{document}
