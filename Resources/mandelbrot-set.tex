\documentclass[paper=a4, fontsize=11pt]{scrartcl}
\usepackage[T1]{fontenc}
%\usepackage{fourier}

\usepackage[english]{babel}
\usepackage[protrusion=true,expansion=true]{microtype}	
\usepackage{amsmath,amsfonts,amsthm} % Math packages
\usepackage[pdftex]{graphicx}	
\usepackage{url}
\usepackage{makecell,pict2e}
\usepackage{comment}
\usepackage{color}
\usepackage{hyperref}
\usepackage{listings}

\lstdefinestyle{mystyle}{
    %backgroundcolor=\color{backcolour},   
    commentstyle=\color{blue},
    keywordstyle=\color{magenta},
    numberstyle=\tiny\color{codegray},
    basicstyle=\footnotesize,
    breakatwhitespace=false,         
    breaklines=true,                 
    captionpos=b,                    
    keepspaces=true,                 
    %numbers=left,                    
    numbersep=5pt,                  
    showspaces=false,                
    showstringspaces=false,
    showtabs=false,                  
    tabsize=2
}
 
\lstset{style=mystyle}
\lstset{language=Python}
\lstset{frame=lines}
%\lstset{label={lst:code_direct}}
\lstset{basicstyle=\footnotesize}

\renewcommand\theadfont{\large}
\newcommand{\gt}[1]{{\color{blue}#1}}
\newcommand{\red}[1]{{\color{red}#1}}

\renewcommand\theadfont{\large}

%%% Custom sectioning
\usepackage{sectsty}
\allsectionsfont{\normalfont\scshape}

\includecomment{answers}
% \excludecomment{answers}

\usepackage[margin=2cm]{geometry}
\setlength{\topmargin}{-2.cm}
\setlength{\headheight}{1cm}

%%% Custom headers/footers (fancyhdr package)
\usepackage{fancyhdr}
\pagestyle{fancyplain}
\fancyhead{}
\fancyfoot[L]{}
\fancyfoot[C]{}
\fancyfoot[R]{\thepage}
\renewcommand{\headrulewidth}{0pt}
\renewcommand{\footrulewidth}{0pt}
\setlength{\headheight}{13.6pt}

%%% Equation and float numbering
\numberwithin{equation}{section}
\numberwithin{figure}{section}
\numberwithin{table}{section}


%%% Maketitle metadata
\newcommand{\horrule}[1]{\rule{\linewidth}{#1}}
\newcommand{\vek}[1]{\mbox{\boldmath $  #1$}}
\newcommand{\ex}[1]{\ensuremath {\mathbb{E}} \left[ #1 \right]}
\newcommand{\var}[1]{\ensuremath{{\rm var}\left[ #1 \right]}}

\title{\usefont{OT1}{bch}{b}{n} \normalfont \normalsize \textsc{SOR3012:
Stochastic Processes} \\ [25pt] \horrule{0.5pt} \\[0.4cm] 
\huge Using random variables to the area of the Mandelbrot set \\
\horrule{2pt} \\[0.25cm]
}
\author{ \normalfont
\normalsize
        Gareth Tribello \\[-3pt] \normalsize
        \today
}
\date{}

\begin{document}
\maketitle

In the blockly exercise you learnt about how we can arrive at an estimate of $\pi$ by generating pairs of uniform random variables. For this portfolio task you should 
use a similar technique to estimate the area of the Mandelbrot set.

The Mandelbrot set contains all the complex numbers, $c$, for which the orbit of 0 under iteration of the quadratic map:
$$
z_{n+1} = z_n^2 + c
$$
remains bounded.  The python script below gives a method for calculating the area of the Mandelbrot set.

\begin{lstlisting}[caption=Calculating the area of the Mandelbrot set]
import numpy as np
import cmath

npoints = 1000
maxiter = 10000

noutside = 0
for i in range(0,npoints) :
    trialp = complex(-2.0 + 2.5*np.random.uniform(0,1),1.125*np.random.uniform(0,1))
    zn1 = trialp
    for j in range(0,maxiter) :
        zn1 = zn1*zn1 + trialp
        r, thet = cmath.polar(zn1)
        if r > 2 :
            noutside = noutside + 1
            break

ninside = npoints - noutside
area = 2.0*(2.5*1.125) * ninside / npoints

print( ninside, area )
\end{lstlisting}

Discuss how this code works and try to extend it so that is also outputs confidence limits on the estimate of the area.  In addition, write some software to generate 
an image of the Mandelbrot set.

\end{document}
