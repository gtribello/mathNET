\documentclass[paper=a4, fontsize=11pt]{scrartcl}
\usepackage[T1]{fontenc}
%\usepackage{fourier}

\usepackage[english]{babel}
\usepackage[protrusion=true,expansion=true]{microtype}	
\usepackage{amsmath,amsfonts,amsthm} % Math packages
\usepackage[pdftex]{graphicx}	
\usepackage{url}
\usepackage{makecell,pict2e}
\usepackage{comment}
\usepackage{color}
\usepackage{hyperref}

\renewcommand\theadfont{\large}
\newcommand{\gt}[1]{{\color{blue}#1}}
\newcommand{\red}[1]{{\color{red}#1}}

\renewcommand\theadfont{\large}

%%% Custom sectioning
\usepackage{sectsty}
\allsectionsfont{\normalfont\scshape}

\includecomment{answers}
% \excludecomment{answers}

\usepackage[margin=2cm]{geometry}
\setlength{\topmargin}{-2.cm}
\setlength{\headheight}{1cm}

%%% Custom headers/footers (fancyhdr package)
\usepackage{fancyhdr}
\pagestyle{fancyplain}
\fancyhead{}
\fancyfoot[L]{}
\fancyfoot[C]{}
\fancyfoot[R]{\thepage}
\renewcommand{\headrulewidth}{0pt}
\renewcommand{\footrulewidth}{0pt}
\setlength{\headheight}{13.6pt}

%%% Equation and float numbering
\numberwithin{equation}{section}
\numberwithin{figure}{section}
\numberwithin{table}{section}


%%% Maketitle metadata
\newcommand{\horrule}[1]{\rule{\linewidth}{#1}}
\newcommand{\vek}[1]{\mbox{\boldmath $  #1$}}
\newcommand{\ex}[1]{\ensuremath {\mathbb{E}} \left[ #1 \right]}
\newcommand{\var}[1]{\ensuremath{{\rm var}\left[ #1 \right]}}

\title{\usefont{OT1}{bch}{b}{n} \normalfont \normalsize \textsc{SOR3012:
Stochastic Processes} \\ [25pt] \horrule{0.5pt} \\[0.4cm] 
\huge Programming the ehrenfest urn problem \\
\horrule{2pt} \\[0.25cm]
}
\author{ \normalfont
\normalsize
        Gareth Tribello \\[-3pt] \normalsize
        \today
}
\date{}

\begin{document}
\maketitle

We can construct a Markov chain by doing the following experiment:

\begin{quotation}
Suppose that you have two cups a pink one and a blue one and that you also have six numbered balls.  Suppose that three of these balls are in the blue cup and that
the remaining three balls are in the pink cup.  You next roll a roll a fair dice to generate a random number $X$ that is between 1 and 6.  Having rolled an $X$ you then
take the $X$th ball and if it is in the blue cup you move it to the pink cup.  If by contrast the $X$th ball is in the pink cup you move it to the blue cup.
\end{quotation}

Draw the transition graph for this Markov chain and write a program to simulate how the number of balls in the blue cup changes with time.  Use your program to calculate
the probablity mass function for the random variable $X(n)$ that tells you how many balls are in the blue cup on step $n$ by calculating a histogram based on the behavior 
of the chain.  Try to calculate confidence limits on the elements of your probablity mass function using resampling. Lastly determine the stationary distribution for this chain 
by using the detailed balance condition.  Compare this analytical result with the numerical result that you obtained by running your program.  

\end{document}
