\documentclass[a4paper]{article}

\usepackage[margin=2.5cm,headheight=50pt,includeheadfoot]{geometry}

\usepackage{amsfonts}
\usepackage{amsmath}
\usepackage{graphicx}
\usepackage{fancyhdr}
\pagestyle{fancy}
\renewcommand{\headrulewidth}{2pt}

\usepackage{xcolor}

\rhead{\includegraphics[width=5cm]{../../html/assets/img/logo.png}}
\lhead{\Huge SOR3012: Tutorial week 11}

\begin{document}

\section{Aim}

The aim of the tutorial this week is to get some critical feedback from your colleagues on the first draft of your report on Markov chains in continuous time.

\section{Bring}

Please bring multiple copies of the first draft of your report on Markov chains in continuous time.  This will be collected in two weeks time but you should already some first attempt of the report 
by now.

\section{Approach}

You will work in groups of three.  Each of you should read the report that was written by each your colleagues. You should then write some feedback on one of your colleagues reports.  Your feedback 
should consist of at least three sentences of continuous prose.  In addition, as you write feedback you should reflect on your own report as you read those of your colleagues and try to determine if 
there is anything that they have done that you should also do.  

You will be given copies of the report descriptions for each of three continuous time Markov chain projects.  Your first point of reference should thus be the markscheme in the project description.  
In fact 
you may choose to determine whether the student has earned all the marks in the report.  In addition, here are some questions you might consider:

\begin{enumerate}
 \item Does the report contain everything that is required by the markscheme?
 \item Is there anything in the report that you do not understand?
 \item Have all the mathematical symbols been used correctly and are there any symbols that have not been defined?
 \item Do the graphs generated by the computer programs behave in the way that you expect?  Remember the results from your numerical experiments should be consistent with the analytical results.
 \item Do you understand the computer programs that have been written?  Should the person writing the report have written more comments in their code?
 \item Are there typographical errors in formulas or equations?
\end{enumerate}

You do not need to write your piece in silence and you can (and should) discuss things with your colleagues before you write them down.  I do, however, have an expectation that you will be able to 
write something about each of the three colleagues in your group during the session and I will be going around to check what you are writing.



\end{document}
